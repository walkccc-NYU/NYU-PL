\documentclass[a4paper,11pt]{article} % This defines the style of your paper

\usepackage[top = 1in, bottom = 0.8in, left = 0.5in, right = 0.5in]{geometry}
\usepackage{float}
\usepackage{fancyhdr}
\usepackage{setspace}
\setlength{\parindent}{0in}
\usepackage{tikz-qtree}

\usepackage{listings}
\usepackage{color}

\definecolor{dark_green}{rgb}{0,0.6,0}
\definecolor{gray}{rgb}{0.5,0.5,0.5}
\definecolor{mauve}{rgb}{0.58,0,0.82}

\usepackage{amsmath}

\lstset{
  frame=none,
  language=Java,
  columns=flexible,
  basicstyle={\small\ttfamily},
  numbers=none,
  numberstyle=\tiny\color{gray},
  keywordstyle=\color{blue},
  commentstyle=\color{dark_green},
  stringstyle=\color{mauve},
  breakatwhitespace=true,
  tabsize=2
}

\newcommand{\code}[1]{\texttt{#1}}

\pagestyle{fancy}
\fancyhf{}

\lhead{\footnotesize Programming Languages Homework 2}
\rhead{\footnotesize Peng-Yu Chen}
\cfoot{\footnotesize \thepage}

\begin{document}
\thispagestyle{empty}

\begin{tabular}{p\linewidth}
{\large \bf Programming Languages - Homework 2} \\ Name: Peng-Yu Chen (pyc305) \\
\hline
\end{tabular}

\vspace{0.4cm}

\begin{enumerate}
  \item % 1.
  \begin{enumerate}
    \item % (a)
    \begin{enumerate}
      \item [(1)] $(\lambda x.E) [M/x] = (\lambda x.E)$
      \item [(2)] $(\lambda x.E) M \underset{\beta}{\Leftrightarrow} E[M/x]$
    \end{enumerate}

    In (1), there are no free occurrences of $x$ in $(\lambda x. E)$, so we
    substitute nothing. \\
    In (2), we substitute $x$ in $E$ with $M$.

    \item % (b)
          $Y = (\lambda h.~
            (\lambda x.~h~(x~x))
            (\lambda x.~h~(x~x))
          )$

    \item % (c)
    \begin{align*}
      Yf &=(\lambda h.~ (\lambda x.~h~(x~x))~(\lambda x.~h~(x~x)))~f
         &\text{definition of $Y$} \\
         &\underset{\beta}{\Leftrightarrow}
          (\lambda x.~f~(x~x))~(\lambda x.~f~(x~x))
         &\text{$\beta$-reduction of $\lambda h$} \\
         &\underset{\beta}{\Leftrightarrow}
          f~((\lambda x.~f~(x~x))~(\lambda x.~f~(x~x)))
         &\text{$\beta$-reduction of $\lambda x$} \\
         &= f~(Yf) &\text{the second equality}
    \end{align*}

    \item % (d)
    \begin{align*}
    \text{Let}~
      \code{fib}~&\text{be}~Yf \\
               f~&\text{be}~(\lambda f.~\lambda x.~\code{if (< x 2) x (+ (f (- x 1)) (f (- x 2)))})
    \end{align*}

    \begin{align*}
    Yf &= f~(Yf) \\
    \code{fib} &= f~\code{fib} \\
    \code{fib 3} &= f~\code{fib 3} \\
    &= (\lambda f.~\lambda x.~\code{if (< x 2) x (+ (f (- x 1)) (f (- x 2)))})~\code{fib}~3 \\
    &\underset{\beta}{\Leftrightarrow} (\lambda x.~\code{if (< x 2) x (+ (fib (- x 1)) (fib (- x 2)))})~3 \\
    &\underset{\beta}{\Leftrightarrow} \code{if (< 3 2) 3 (+ (fib (- 3 1)) (fib (- 3 2)))} \\
    &\underset{\delta}{\Leftrightarrow} \code{+ (fib 2) (fib 1)}
    \end{align*}

    \item % (e)
    Church-Rosser Theorem I: if two expressions $E_1$ and $E_2$ are interconvertible,
    then there is an expression $E$, which can be converted from both $E_1$ and $E_2$. \\
    By having the above property, we can conclude that there is no way a single expression
    can be converted to two different normal forms. \\

    Church-Rosser Theorem II: if an expression $E_1$ can be converted to a normal
    form expression $E_2$, then we can use normal order reduction to convert $E_1$ to $E_2$.

  \end{enumerate}

  \item % 2.
  \begin{enumerate}
    \item % (a)
    % ('a -> 'b) -> ('b list -> 'c list) -> ('c -> 'd) -> 'a -> 'd
    %   f             g                       h            a
    %
    % use map to constraint that (g [f a]): 'c list
    % also, since h is just after map, so h: 'c -> 'd
    \code{fun foo f g h a = hd (map h (g [f a]))}

    \item % (b)
    % sml:(int list -> 'a) -> ('b * 'c -> bool) -> 'b -> int list * 'c -> 'a
    \code{(int list -> 'c) -> ('a * 'b -> bool) -> 'a -> int list * 'b -> 'c}

    \item % (c)
    \code{x}'s type is unconstraint, so let's assume its type is \code{'a}. \\
    \code{z}'s type is unconstraint, so let's assume its type is \code{'b}. \\
    \code{w}'s type is constraint since \code{bar} applies to \code{w} and also applies
    to \code{[1, 2, 3]}, \code{w}'s type is \code{int list}. \\
    \code{y}'s type is constraint by \code{if x > z then y else w}, \code{y}'s type
    should be same as \code{w}'s type \code{int list}. \\
    \code{bar}'s type is constraint by \code{bar w = if x > z then y else w}, so its type
    is \code{int list -> int list}. \\
    \code{(op >)}'s type is constraint by \code{x > z}, so its type should be
    \code{('a * 'b -> bool)} \\
    \code{f}'s input type is constraint by \code{f (bar [1, 2, 3])}, which
    is \code{int list}. \\
    \code{f}'s output type is unconstraint, so let's assume it's \code{'c}. \\
    Combine the parameters' types of \code{foo} together, we have:

    \begin{itemize}
      \item \code{f:~(int list -> 'c)}
      \item \code{(op >):~('a * 'b -> bool)}
      \item \code{x:~'a}
      \item \code{(y, z):~int list * 'b}
      \item return type is \code{'c}
    \end{itemize}
  \end{enumerate}

  \item % 3.
  \begin{enumerate}
    \item % (a)
    \begin{enumerate}
      \item Encapsulation of data with the functions that operate on that data
      \item Inheritance
      \item Subtyping with dynamic dispatch
    \end{enumerate}
    \item % (b)
    Assume we have a class \code{Vehicle} and a class \code{Car} which inherits
    from the class \code{Vehicle}.

    But, a \code{Car} object may have more fields than a \code{Vehicle} object.

    We know that the set defined by a type (class) \code{T} is the set of all
    objects that have the properties of \code{T}, and may have additional properties.

    Since all objects of \code{Car} have the properties of \code{Vehicle}, and
    may have additional properties, all objects in \code{Car} set are also in
    \code{Vehicle} set. So, the set of \code{Car} $\subseteq$ the set
    of \code{Vehicle}.

    \item % (c)
    Assume we have the following two classe \code{Vehicle} and \code{Car}:

\begin{lstlisting}
class Vehicle {
  int speed;
}
\end{lstlisting}

\begin{lstlisting}
class Car extends Vehicle {
  String model;
}
\end{lstlisting}

    Since all objects of \code{Car} have the properties of \code{Vehicle}, and
    have an additional property \code{model}, all objects in \code{Car} set are
    also in \code{Vehicle} set. So, the set of \code{Car} $\subseteq$ the set of
    \code{Vehicle}.

    \item % (d)
    \begin{enumerate}
      \item [(1)] Suppose we have a class \code{A} and a class \code{B} which
      inherits from the class \code{A}. We cannot call \code{f(h)} because
      \code{h} gets called on an \code{A} object. That's not allowed since an
      \code{A} object is not a \code{B} object. In the following example, we
      access \code{b.y} which is a field that only appears in \code{B} and not in
      \code{A}.

\begin{lstlisting}
object CovariantDemo extends App {
  class A {
    val x = 1
  }

  class B extends A {
    val y = 2
  }

  def f(g: A => Int) = g(new A)
  def h(b: B) = b.y  // access a field that only appears in B and not in A

  f(h)  // not allowed because an A object is not a B object
}
\end{lstlisting}

      \item [(2)] Suppose we have a class \code{A} and a class \code{B} which
      inherits from the class \code{A}. In the following example, calling
      \code{g(3)} is actually calling \code{h(3)} which returns an \code{A}
      object. However, we cannot assign an \code{A} object to type \code{B}.

\begin{lstlisting}
object ContravariantDemo extends App {
  class A
  class B extends A

  def f(g: Int => B): Unit = {
    val b: B = g(3) // actually calling h(3) which returns an A object
  }
  def h(x: Int) = new A

  f(h)  // not allowed because we cannot assign an A object to type B
}
\end{lstlisting}

    \end{enumerate}
    \item % (e)
    Suppose we have a class \code{A} and a class \code{B} which inherits from
    the class \code{A}.

    \begin{enumerate}
      \item [(1)]
      We know that \code{A => Int} contains the set of all functions that can be
      called on an \code{A} object and return an \code{Int}. For example,
      \code{def f(a:~A) = 1} is in \code{A => Int}.

      Also, any functions in \code{A => Int} can be called on a \code{B}
      object and is therefore in \code{B => Int}.

      Therefore, \code{A => Int} $\subseteq$ \code{B => Int}.
      \item [(2)]
      Consider \code{def f(g:~Int) = new B}, the set of all functions that can
      be called on an \code{Int} and return a \code{B} object.

      Since any function returning a \code{B} is returning an \code{A}, all
      functions in \code{Int => B} are in \code{Int => A}.

      Therefore \code{Int => B} $\subseteq$ \code{Int => A}.
    \end{enumerate}
  \end{enumerate}

  \item % 4.
  \begin{enumerate}
    \item % (a)
    Suppose the compiler allows covariant subtyping among instances of a generic
    class. Assume \code{ArrayList<B>} is a subtype of \code{ArrayList<A>}:

\begin{lstlisting}
void addA(ArrayList<A> al) {
  al.add(new A());
}

ArrayList<B> bl = new ArrayList<>();
addA(bl);          // assume ArrayList<B> is a subtype of ArrayList<A>
B b = bl.get(0); // error, returns an A, which is not a B
\end{lstlisting}

    \item % (b)

\begin{lstlisting}
void addA(ArrayList<? super A> al) {
  al.add(new A()); // it's okay to add an A to an ArrayList of elements of
                     // A's supertype (including A)
}
\end{lstlisting}

  \end{enumerate}

  \item % 5.
  \begin{enumerate}
    \item % (a)
\begin{lstlisting}
abstract class Tree[T <: Ordered[T]]
case class Node[T <: Ordered[T]](v: T, l: Tree[T], r: Tree[T]) extends Tree[T]
case class Leaf[T <: Ordered[T]](v: T) extends Tree[T]

class OrderedInt(x: Int) extends Ordered[OrderedInt] {
  def get = x
  def compare(that: OrderedInt) = x - that.get
  override def toString(): String = (s"$x")
}

object Prog {
  def min[T <: Ordered[T]](tree: Tree[T]): T = tree match {
    case Node(value, left, right) =>
      val minL = min(left)
      val minR = min(right)
      val minChild = if (minL < minR) minL else minR
      return if (value < minChild) value else minChild
    case Leaf(value) => value
  }

  def main(args: Array[String]) = {
    val x: Tree[OrderedInt] = Node(
      new OrderedInt(-1),
      Node(new OrderedInt(0), Leaf(new OrderedInt(3)), Leaf(new OrderedInt(5))),
      Node(new OrderedInt(8), Leaf(new OrderedInt(7)), Leaf(new OrderedInt(1)))
    )

    print(min(x)) // should print -1
  }
}
\end{lstlisting}

    \item % (b)
\begin{lstlisting}
class A(x: Int) {
  override def toString() = "A<" + x + ">"
}

class B(x: Int, y: Int) extends A(x) {
  override def toString() = "B<" + x + "," + y + ">"
}

class C[+E](l: List[E]) {
  override def toString() = (s"$l")
}
\end{lstlisting}

    \item % (c)
\begin{lstlisting}
object CovariantDemo {
  def printC(l: C[A]) = {
    println(l)
  }

  def main(args: Array[String]) = {
    val listA = new C(List(new A(1)))
    val listB = new C(List(new B(1, 2)))

    printC(listA)
    printC(listB) // uses covariant subtyping (C[B] <: C[A])
  }
}
\end{lstlisting}

    \item % (d)
\begin{lstlisting}
class A(x: Int) {
  override def toString() = "A<" + x + ">"
}

class B(x: Int, y: Int) extends A(x) {
  override def toString() = "B<" + x + "," + y + ">"
}

class C[-E]() {
  var l: List[Any] = List()
  def insert(x: E) = l = x :: l

  override def toString() = (s"$l")
}
\end{lstlisting}

    \item % (e)
\begin{lstlisting}
object ContravariantDemo {
  def insertC(l: C[B], b: B) =
    l.insert(b)

  def main(args: Array[String]) = {
    val listB = new C[B]()
    val listA = new C[A]()

    insertC(listB, new B(1, 2))
    insertC(listA, new B(3, 4)) // uses contravariant subtyping (C[A] <: C[B])
    listA.insert(new A(5))

    println(listB)
    println(listA)
  }
}
\end{lstlisting}

  \end{enumerate}

  \item % 6.
  \begin{enumerate}
    \item % (a)

    Mark-and-sweep garbage collector (GC) won't be affected if there's a cycle,
    while reference counting collector might need a backup mark-and-sweep collector
    when the heap fills up with cycles.

    \item % (b)

    \begin{itemize}
      \item Copying GC compacts the live objects, so we could use a heap pointer
      for faster storage allocation.
      \item Unlike mark-and-sweep GC, the cost of copying GC is proportional to
      the amount of the live objects, not the size of the heap.
    \end{itemize}

    \item % (c)

    Generational copying GC uses many heaps instead of two heaps.
    Each heap contains objects of similar age. These heaps are called generations,
    for  obvious reasons. When a heap fills up, the live objects in that heap is
    copied to the next (older) heap.

    When a heap fills up, the generational copying GC copies the live objects in
    only that heap to the next older heap, so no other heap is touched during GC.

    The advantages of generational copying GC over the simple copying GC are:

    \begin{itemize}
      \item Since the youngest heap doesn't contain all the older objects, rather
      only the youngest objects, GC will be triggered less frequently.

      \item Since young objects tend to die at a higher rate than older objects,
      and only live objects are copied to the older heap, GC will take less time
      on the youngest heap.

      \item GC will rarely happen on the older heaps which contain most of the
      live objects.
    \end{itemize}

    \item % (d)

\begin{lstlisting}
void delete(Object* x) {
  x->refcount = x->refcount - 1;
  if (x->refcount == 0) {
    for (Object* child : x->children) {
      delete(child);
    }
    addToFreeList(x);
  }
}
\end{lstlisting}
  \end{enumerate}

\end{enumerate}

\end{document}
